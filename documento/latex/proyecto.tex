%********************************************************************************************************
%  Tipo de documento
%********************************************************************************************************
\documentclass[a4paper,11pt]{article}
%********************************************************************************************************
%  Paquetes empleados
%********************************************************************************************************
\usepackage[utf8]{inputenc} % Esto es para poder escribir acentos directamente
\usepackage[spanish]{babel} % Esto es para que el LaTeX sepa que el texto está en español
\usepackage{amsmath, amsthm, amsfonts} % Paquetes de la AMS
\usepackage{fancyhdr}
\usepackage{enumerate} % Paquete de enumeraciones
\usepackage{hyperref}
\usepackage{graphicx}
\usepackage{colortbl}

%********************************************************************************************************
\title
{
    \textbf{Práctica XII}\\
    \textbf{Proyecto}
}
\author
{
 	\textbf{Raúl Arroyo Lubián}\\
    \textbf{Alfonso Lacalle García}\\
    \textbf{Pedro Daniel Lopez Gonzalez}
}

% Renombrando
\renewcommand\tablename{Tabla}
\renewcommand\figurename{Figura}
\renewcommand\contentsname{Tabla de contenidos}
\renewcommand\refname{Bibliografí­a}

\pagestyle{fancy}
\fancyhf{}
% En lo siguiente, fancyhead sirve para configurar la cabecera, fancyfoot para el pie.
% Justificación: C=centered, R=right, L=left, (nada)=LRC
% Página: O=odd, E=even, (nada)=OE
\fancyhead[C]{\rightmark}
\fancyhead[C]{\leftmark}
\fancyfoot[C]{\thepage}

% Modifica el ancho de las lí­neas de cabecera y pie
\renewcommand{\headrulewidth}{0.4pt}
\renewcommand{\footrulewidth}{0.4pt}
\setlength{\headheight}{1.5\headheight} % Aumenta la altura del encabezado en una vez y media

%********************************************************************************************************
\begin{document}
	%****************************************************************************************************
	% Portada
	%****************************************************************************************************
	\maketitle
	\newpage
	%****************************************************************************************************
	% Indice
	%****************************************************************************************************
	\tableofcontents
	\pagenumbering{arabic}
	\newpage
	%****************************************************************************************************
	% Contenido
	%****************************************************************************************************
	\section{Sección}
    
\end{document}
%********************************************************************************************************