%********************************************************************************************************
%  Tipo de documento
%********************************************************************************************************
\documentclass[a4paper,11pt]{article}
%********************************************************************************************************
%  Paquetes empleados
%********************************************************************************************************
\usepackage[utf8]{inputenc} % Esto es para poder escribir acentos directamente
\usepackage[spanish]{babel} % Esto es para que el LaTeX sepa que el texto está en español
\usepackage{amsmath, amsthm, amsfonts} % Paquetes de la AMS
\usepackage{fancyhdr} % Paquete de cabeceras y pie de páginas bonitos
\usepackage{enumerate} % Paquete de enumeraciones
\usepackage{hyperref}
\usepackage{graphicx}
\usepackage{colortbl}

%********************************************************************************************************
% Renombrando
\renewcommand\tablename{Tabla}
\renewcommand\figurename{Figura}
\renewcommand\contentsname{Tabla de contenidos}
\renewcommand\refname{Bibliografí­a}

\pagestyle{fancy}
\fancyhf{}
% En lo siguiente, fancyhead sirve para configurar la cabecera, fancyfoot para el pie.
% Justificación: C=centered, R=right, L=left, (nada)=LRC
% Página: O=odd, E=even, (nada)=OE
\fancyhead[C]{\rightmark}
\fancyhead[C]{\leftmark}
\fancyfoot[C]{\thepage}

% Modifica el ancho de las lí­neas de cabecera y pie
\renewcommand{\headrulewidth}{0.4pt}
\renewcommand{\footrulewidth}{0.4pt}
\setlength{\headheight}{1.5\headheight} % Aumenta la altura del encabezado en una vez y media

%********************************************************************************************************
\begin{document}
	%****************************************************************************************************
	% Portada
	%****************************************************************************************************
	\begin{titlepage}
		\begin{center}
			\includegraphics[width=375px]{Universidad.png} 
            \includegraphics[width=275px]{logo_cordosoft.png}
			\textbf{\LARGE Documento de especificación}\\
			\textbf{\Large Sistema de diagnóstico}\\
			\textbf{Restauración/Conservación de pintura al temple al huevo sobre tabla
			sin entelar}\\
			\\
			\textbf{}
			\\
			\textbf{Autores}
			\\
			\textbf{}
			\\
			\begin{tabular}{l l}
				\textbf{Raul Arroyo Lubián} & \textbf{i02arlur@uco.es} \\	
				\textbf{Pedro Daniel López González} & \textbf{i02logop@uco.es} \\
				\textbf{Alfonso Lacalle García} & \textbf{i52lagaar@uco.es} \\
			\end{tabular}
			\\
			\textbf{}
			\\
			\textbf{}
			\\
			\begin{tabular}{l l}
				\textbf{Fecha de creación:} & \textbf{17 de mayo de 2013} \\
				\textbf{Fecha de última modificación:} & \textbf{\today } \\
			\end{tabular}
		\end{center}
    \end{titlepage}
	\newpage
	%****************************************************************************************************
	% Indice
	%****************************************************************************************************
	\tableofcontents
	\pagenumbering{arabic}
	\newpage
	%****************************************************************************************************
	% Contenido
	%****************************************************************************************************
	\section{Sección}
    Hola
\end{document}
%********************************************************************************************************