%********************************************************************************************************
%  Tipo de documento
%********************************************************************************************************
\documentclass[a4paper,11pt]{article}
%********************************************************************************************************
%  Paquetes empleados
%********************************************************************************************************
\usepackage[utf8]{inputenc} % Esto es para poder escribir acentos directamente
\usepackage[spanish]{babel} % Esto es para que el LaTeX sepa que el texto está en español
\usepackage{amsmath, amsthm, amsfonts} % Paquetes de la AMS
\usepackage{fancyhdr}
\usepackage{enumerate} % Paquete de enumeraciones
\usepackage{hyperref}
\usepackage{graphicx}
\usepackage{colortbl}
\usepackage{pdflscape}

% Renombrando
\renewcommand\tablename{Tabla}
\renewcommand\figurename{Figura}
\renewcommand\contentsname{Tabla de contenidos}
\renewcommand\refname{Bibliografía}

\pagestyle{fancy}
\fancyhf{}
% En lo siguiente, fancyhead sirve para configurar la cabecera, fancyfoot para el pie.
% Justificación: C=centered, R=right, L=left, (nada)=LRC
% Página: O=odd, E=even, (nada)=OE
\fancyhead[C]{\rightmark}
\fancyhead[C]{\leftmark}
\fancyfoot[C]{\thepage}

% Modifica el ancho de las lí­neas de cabecera y pie
\renewcommand{\headrulewidth}{0.4pt}
\renewcommand{\footrulewidth}{0.4pt}
\setlength{\headheight}{1.5\headheight} % Aumenta la altura del encabezado en una vez y media


%********************************************************************************************************
\begin{document}
    %***************************************************************************************************
	%  Portada
	%***************************************************************************************************
	\begin{titlepage}
		\begin{center}
			\includegraphics[width=375px]{Universidad.png} \\
            \includegraphics[width=275px]{logo_cordosoft.png} \\
			\textbf{\LARGE Documento de especificación} \\
			\textbf{\Large SDJS - RCPTTE} \\
			\\
			\textbf{}
			\\
			\textbf{Autores}
			\\
			\textbf{}
			\\
			\begin{tabular}{l l}
				\textbf{Raul Arroyo Lubián} & \textbf{i02arlur@uco.es} \\	
				\textbf{Pedro Daniel López González} & \textbf{i02logop@uco.es} \\
				\textbf{Alfonso Lacalle García} & \textbf{i52lagaa@uco.es} \\
			\end{tabular}
			\\
			\textbf{}
			\\
			\textbf{}
			\\
			\begin{tabular}{l l}
				\textbf{Fecha de creación:} & \textbf{17 de mayo de 2013} \\
				\textbf{Fecha de última modificación:} & \textbf{\today } \\
			\end{tabular}
		\end{center}
    \end{titlepage}
	\newpage
	%***************************************************************************************************
	%  Indice
	%***************************************************************************************************
	\tableofcontents
	\pagenumbering{arabic}
	\newpage
	%***************************************************************************************************
	% Contenido
	%***************************************************************************************************
	\section{Introducción}
	\section{Modelado de contexto}
		\subsection{Modelo de organización}
			\subsubsection{Formulario OM-1}
			\begin{center}
				\begin{tabular}{| l | l |}
					\hline
					\textbf{Problemas y oportunidades} & \\
					\hline
					\textbf{Contexto organizacional} & \\
					\hline
					\textbf{Soluciones} & \\
					\hline
				\end{tabular}
			\end{center}
			\subsubsection{Formulario OM-2}
			Incluye un diagrama de la empresa
			\begin{center}
				\begin{tabular}{| l | l |}
					\hline
					\textbf{Estructura} & \\
					\hline
					\textbf{Procesos} & \\
					\hline
					\textbf{Personal} & \\
					\hline
					\textbf{Recursos} & \\
					\hline
					\textbf{Conocimiento} & \\
					\hline
					\textbf{Cultura y potencial} & \\
					\hline
				\end{tabular}
			\end{center}
			\newpage
			\begin{landscape}
			\subsubsection{Formulario OM-3}
			\begin{center}
				\begin{tabular}{| l | l | p{2.8cm} | p{2cm} | p{5cm} | p{3cm} | p{2.5cm} |}
					\hline
					\textbf{Nº} & \textbf{Tarea} & \textbf{Realizada por} & \textbf{¿Donde?} & \textbf{Recursos de conocimiento} &
					\textbf{¿Intensivo en conocimiento?} & \textbf{Importancia}\\
					\hline
					 & & & & & &\\
					\hline
				\end{tabular}
			\end{center}
			\end{landscape}
			\newpage
			\begin{landscape}
			\subsubsection{Formulario OM-4}
			\begin{center}
				\begin{tabular}{| p{5cm} | p{2.4cm} | p{2cm} | p{2cm} | p{2cm} | p{2cm} |
				p{2cm} |}
					\hline
					\textbf{Recurso de conocimiento} & \textbf{Pertenece a} & \textbf{Usado en} & \textbf{¿Forma correcta?} & \textbf{¿Lugar correcto?} &
					\textbf{¿Tiempo correcto?} & \textbf{¿Calidad concreta?}\\
					\hline
					 & & & & & &\\
					\hline
				\end{tabular}
			\end{center}
			\end{landscape}
			\subsubsection{Formulario OM-5}
			\begin{center}
				\begin{tabular}{| l | l |}
					\hline
					\textbf{Viabilidad empresarial} & \\
					\hline
					\textbf{Viabilidad técnica} & \\
					\hline
					\textbf{Viabilidad del proyecto} & \\
					\hline
					\textbf{Acciones propuestas} & \\
					\hline
				\end{tabular}
			\end{center}
		\subsection{Modelo de tareas}
			\subsubsection{Formulario TM-1}
			Hay que hacer uno por cada tarea
			\begin{center}
				\begin{tabular}{| l | l |}
					\hline
					\textbf{Tarea} & \\
					\hline
					\textbf{Organización} & \\
					\hline
					\textbf{Objetivo y valor} & \\
					\hline
					\textbf{Dependencias y flujos} & \\
					\hline
					\textbf{Objetos manipulados} & \\
					\hline
					\textbf{Tiempo y control} & \\
					\hline
					\textbf{Agentes} & \\
					\hline
					\textbf{Conocimiento y capacidad} & \\
					\hline
					\textbf{Recursos} & \\
					\hline
					\textbf{Calidad y eficiencia} & \\
					\hline
				\end{tabular}
			\end{center}
			\subsubsection{Formulario TM-2}
			Hay que hacer uno por cada elemento de conocimiento usado en una tarea
			\begin{center}
				\begin{tabular}{| l | l |}
					\hline
					\textbf{Nombre} & \\
					\hline
					\textbf{Poseído por} & \\
					\hline
					\textbf{Usado en} & \\
					\hline
					\textbf{Dominio} & \\
					\hline
				\end{tabular}
				\\
				\textbf{}
				\\
				\begin{tabular}{| p{6.3cm} | l | p{3.8cm} |}
					\hline
					\textbf{Naturaleza del conocimiento} & \textbf{(si/no)} & \textbf{¿Cuello
					de botella?}\\
					\hline
					\textbf{Forma, riguroso} & & \\
					\hline
					\textbf{Empírico, cuantitativo} & & \\
					\hline
					\textbf{Específico del dominio} & & \\
					\hline
					\textbf{Basado en la experiencia} & & \\
					\hline
					\textbf{Basado en la acción} & & \\
					\hline
					\textbf{Incompleto} & & \\
					\hline
					\textbf{Incierto, puede ser incorrecto} & & \\
					\hline
					\textbf{Cambia con rapidez} & & \\
					\hline
					\textbf{Difícil de verificar} & & \\
					\hline
					\textbf{Tácito, difícil de transferir} & & \\
					\hline
					\textbf{Forma de conocimiento} & \textbf{(si/no)} & \textbf{¿Cuello
					de botella?}\\
					\hline
					\textbf{Mental} & & \\
					\hline
					\textbf{Papel} & & \\
					\hline
					\textbf{Electrónica} & & \\
					\hline
					\textbf{Habilidades} & & \\
					\hline
					\textbf{Otros} & & \\
					\hline
					\textbf{Disponibilidad del conocimiento} & \textbf{(si/no)} & \textbf{¿Cuello
					de botella?}\\
					\hline
					\textbf{Limitaciones en tiempo} & & \\
					\hline
					\textbf{Limitaciones en espacio} & & \\
					\hline
					\textbf{Limitaciones en acceso} & & \\
					\hline
					\textbf{Limitaciones en calidad} & & \\
					\hline
					\textbf{Limitaciones en forma} & & \\
					\hline
				\end{tabular}
			\end{center}
		\subsection{Modelo de agentes}
			\subsubsection{Formulario AM-1}
			Hay que crear uno por cada agente
			\begin{center}
				\begin{tabular}{| l | l |}
					\hline
					\textbf{Nombre} & \\
					\hline
					\textbf{Organización} & \\
					\hline
					\textbf{Implicado en} & \\
					\hline
					\textbf{Se comunica con} & \\
					\hline
					\textbf{Conocimiento} & \\
					\hline
					\textbf{Otras competencias} & \\
					\hline
					\textbf{Responsabilidad y restricciones} & \\
					\hline
				\end{tabular}
			\end{center}
		\subsection{Formulario resumen}
			\subsubsection{Formulario OTA-1}
			\begin{center}
				\begin{tabular}{| l | l |}
					\hline
					\textbf{Impactos y cambios en la organización} & \\
					\hline
					\textbf{Impactos y cambios en tareas y agentes} & \\
					\hline
					\textbf{Actitudes y compromisos} & \\
					\hline
					\textbf{Acciones propuestas} & \\
					\hline
				\end{tabular}
			\end{center}
	\section{Modelado conceptual}
		\subsection{Modelo de conocimiento}
			\subsubsection{Formulario KM-1}
		\subsection{Modelo de comunicación}
			\subsubsection{Formulario CM-1}
			Hay que hacer uno por cada transacción, quizás
			\begin{center}
				\begin{tabular}{| l | l |}
					\hline
					\textbf{Nombre de transacción} & \\
					\hline
					\textbf{Objetos de información} & \\
					\hline
					\textbf{Agentes involucrados} & \\
					\hline
					\textbf{Plan de comunicaciones} & \\
					\hline
					\textbf{Restricciones} & \\
					\hline
					\textbf{Especificaciones del intercambio de información} & \\
					\hline
				\end{tabular}
			\end{center}
			\subsubsection{Formulario CM-2}
			Hay que hacer uno por cada transacción, quizás
			\begin{center}
				\begin{tabular}{| l | l |}
					\hline
					\textbf{Transacción} & \\
					\hline
					\textbf{Agentes involucrados} & \\
					\hline
					\textbf{Ítems de información} & \\
					\hline
					\textbf{Especificaciones de los mensajes} & \\
					\hline
					\textbf{Control sobre mensajes} & \\
					\hline
				\end{tabular}
			\end{center}
	\section{Modelado de diseño}
		\subsection{Modelo de diseño}
			\subsubsection{Formulario DM-1}
			\begin{center}
				\begin{tabular}{| l | l |}
					\hline
					\textbf{Decisiones arquitectónicas} & \textbf{Formato} \\
					\hline
					\textbf{Organización de los subsistemas} & \\
					\hline
					\textbf{Modelo de control} & \\
					\hline
					\textbf{Descomposición de subsistemas} & \\
					\hline
				\end{tabular}
			\end{center}
			\subsubsection{Formulario DM-2}
			\begin{center}
				\begin{tabular}{| l | l |}
					\hline
					\textbf{Producto software} &  \\
					\hline
					\textbf{Hardware potencial} & \\
					\hline
					\textbf{Hardware de desarrollo} & \\
					\hline
					\textbf{Librería de visualización} & \\
					\hline
					\textbf{Lenguaje de implementación} &  \\
					\hline
					\textbf{Representación del conocimiento} & \\
					\hline
					\textbf{Protocolos de interacción} & \\
					\hline
					\textbf{Control de flujo} & \\
					\hline
					\textbf{Soporte para CommonKADS} & \\
					\hline
				\end{tabular}
			\end{center}
			\subsubsection{Formulario DM-3}
			\begin{center}
				\begin{tabular}{| l | l |}
					\hline
					\textbf{Elemento de la arquitectura} & \textbf{Elementos típicos de
					decisión} \\
					\hline
					\textbf{Controlador} & \\
					\hline
					\textbf{Tarea} & \\
					\hline
					\textbf{Método de la tarea} & \\
					\hline
					\textbf{Inferencia} &  \\
					\hline
					\textbf{Método de la inferencia} & \\
					\hline
					\textbf{Rol estático} & \\
					\hline
					\textbf{Rol dinámico} & \\
					\hline
					\textbf{Base de conocimiento} & \\
					\hline
					\textbf{Vistas} & \\
					\hline
				\end{tabular}
			\end{center}
			\subsubsection{Formulario DM-4}
			\begin{center}
				\begin{tabular}{| l | l | l |}
					\hline
					\textbf{Elemento} & \textbf{Decisión de diseño} & \textbf{Comentarios} \\
					\hline
					\textbf{Controlador} &  & \\
					\hline
					\textbf{Método de la tarea} &  & \\
					\hline
					\textbf{Roles dinámicos} &  & \\
					\hline
					\textbf{Inferencias} &  & \\
					\hline
					\textbf{Métodos de las inferencias} &  & \\
					\hline
					\textbf{Bases de conocimiento} &  & \\
					\hline
					\textbf{Vistas de los objetos} &  & \\
					\hline
				\end{tabular}
			\end{center}
\end{document}